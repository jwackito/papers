\documentclass[final,narroweqnarray,inline,twoside]{ieee}

\usepackage{ucs}
\usepackage[utf8x]{inputenc}
\usepackage[spanish]{babel}
\usepackage{fontenc}
\usepackage{graphicx}
\usepackage{textcomp}
% \usepackage{tipa}

\newcommand{\itref}[1]{[{#1}]}
%uso: \itref{1}
%1 es el número de la referencia

\author{Joaquín Bogado\\ Andrea Gómez del Mónaco
\\LINTI. Facultad de Informática, Universidad Nacional de La Plata.
\\La Plata, B1900ASD, ARGENTINA
}
\title{Modificando lxrandr para que guarde los cambios al reiniciar}

\hyphenation{}
\begin{document}
\pagenumbering{none}
\maketitle
\sloppy

\begin{abstract}
En entornos LXDE, la aplicación lxrandr sirve para ajustar la tasa de refresco y la resolución del monitor. Esta herramienta
es una interfaz gráfica para el programa xrandr y no permite guardar las configuraciones y los cambios se pierden una vez
que se reinicia la sesión. Este trabajo incluye una posible modificación al programa para quelas modificaciones sean
permanentes entre inicios de sesión.
\end{abstract}

\noindent \textbf{Keywords: } lxde, lxrandr, xrandr, configuración del monitor

\section{Introducción}
LXDE es un entorno gráfico liviano para máquinas de escasos recursos que corre sobre el sistema Xwindow. LXDE provee una
interfaz gráfica para el comando xrandr llamada lxrandr que permite cambiar tanto la resolución como la tasa de refresco del
monitor en tiempo real. El programa es muy sencillo y esta escrito en C haciendo uso de las librerías gráficas GTK. Debido
su simplicidad, no contempla la posibilidad de guardar las configuraciones. Así, para forzar una configuración a 800x600 hay
que correr el programa cada vez. Este trabajo explica como modificar el código de lxrandr para que los cambios sean
permanentes entre inicios de sesión. 

\section{Funcionamiento de lxrandr}
\texttt{lxrandr} hace un llamado a xrandr para pedir los modos (tanto resoluciones como tasas de refresco) soportados por la
placa de video y el monitor mediante la función \texttt{nombre\_de\_la\_función()}. Una vez cargados los modos, los muestra
a través de un menú contextual. Una vez que se selecciona el modo deseado, cuando se presiona el boton aceptar, se llama a
la función \texttt{nombre\_de\_la\_funcion\_para\_setear\_los\_modos()} que invoca al comando xrandr con los parámetros
adecuados para setear ese modo. De esta manera, al iniciar de nuevo la sesión, los cambios no se ven reflejados debido a que
se cargan los modos por defecto definidos para Xorg.

\section{Guardando los cambios}
La idea de este trabajo es modificar la función \texttt{nombre\_de\_la\_funcion\_para\_setear\_los\_modos()} de forma tal
que guarde los cambios de los modos seleccionados.
\section{Casos de prueba}

\section{Trabajo a futuro}

\section{Conclusiones}

\section{Agradecimientos}

\section{Referencias y enlaces de interés}

\begin{enumerate}
\item Página principal de Lihuen GNU/Linux\\
\newblock {\texttt{http://lihuen.linti.unlp.edu.ar/index.php /Portada}}\\[Visitado el día 9 de enero de 2009]
\end{enumerate}

\end{document}