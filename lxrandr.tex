\documentclass[final,narroweqnarray,inline,twoside]{ieee}

\usepackage{ucs}
\usepackage[utf8x]{inputenc}
\usepackage[spanish]{babel}
\usepackage{fontenc}
\usepackage{graphicx}
\usepackage{textcomp}
% \usepackage{tipa}

\newcommand{\itref}[1]{[{#1}]}
%uso: \itref{1}
%1 es el número de la referencia

\author{Joaquín Bogado\\ Andrea Gómez del Mónaco
\\LINTI. Facultad de Informática, Universidad Nacional de La Plata.
\\La Plata, B1900ASD, ARGENTINA
}
\title{Modificando lxrandr para que guarde los cambios al reiniciar}

\hyphenation{}
\begin{document}
\pagenumbering{none}
\maketitle
\sloppy

\begin{abstract}
En entornos LXDE, la aplicación lxrandr sirve para ajustar la tasa de refresco y la resolución del monitor. Esta herramienta
es una interfaz gráfica para el programa xrandr y no permite guardar las configuraciones y los cambios se pierden una vez
que se reinicia la sesión. Este trabajo incluye una posible modificación al programa para quelas modificaciones sean
permanentes entre inicios de sesión.
\end{abstract}

\noindent \textbf{Keywords: } lxde, lxrandr, xrandr, configuración del monitor

\section{Introducción}
LXDE es un entorno gráfico liviano para máquinas de escasos recursos que corre sobre el sistema Xwindow. LXDE provee una
interfaz gráfica para el comando xrandr llamada lxrandr que permite cambiar tanto la resolución como la tasa de refresco del
monitor en tiempo real. El programa es muy sencillo y esta escrito en C haciendo uso de las librerías gráficas GTK. Debido
su simplicidad, no contempla la posibilidad de guardar las configuraciones. Así, para forzar una configuración a 800x600 hay
que correr el programa cada vez. Este trabajo explica como modificar el código de lxrandr para que los cambios sean
permanentes entre inicios de sesión. 

\section{Funcionamiento de lxrandr}
\texttt{lxrandr} hace un llamado a xrandr para pedir los modos (tanto resoluciones como tasas de refresco) soportados por la
placa de video y el monitor mediante la función \texttt{get\_xrandr\_info()}. Una vez cargados los modos, los muestra
a través de un menú contextual. Una vez que se selecciona el modo deseado, cuando se presiona el botón aceptar, se llama a
la función \texttt{set\_xrandr\_info()} que invoca al comando xrandr con los parámetros adecuados para setear ese modo.
al iniciar de nuevo la sesión, los cambios no se ven reflejados debido a que se cargan los modos por defecto definidos
para Xorg.

\section{Guardando los cambios}
La idea de este trabajo es modificar la función \texttt{set\_xrandr\_info()} de forma tal que guarde los cambios de los modos
seleccionados cuando se presiona el botón aceptar. Esto se puede lograr guardando el comando xrandr con sus argumentos en un
script que se ejecute cada inicio de sesión. Este script estará guardado la carpeta personal de usuario. De esta manera:
\begin{enumerate}
 \item Los cambios son propios del usuario. Diferentes usuarios pueden setear diferentes configuraciones para la pantalla.
 \item No es necesario tener privilegios de administrador para cambiar la resolución de la pantalla.
\end{enumerate}


\section{Primera propuesta}
Los cambios propuestos para la función \texttt{set\_xrandr\_info()} detectan la ubicación de la carpeta home mediante el uso
de la función getenv(). Cada vez que se inicia sesión, el archivo \texttt{\$HOME/.profile} es invocado. Una posible solución
sería bajar el contenido de la variable \texttt{cmd->str} directamente al archivo \texttt{\$HOME/.profile}. 

El problema de esta solución es que cada vez que se llame a la función \texttt{set\_xrandr\_info()} se agregará una llamada
a xrandr, muy posiblemente con valores distintos cada vez. Para evitar este inconveniente se puede realizar una busqueda en
el archivo \texttt{\$HOME/.profile} por una llamada a xrandr y sobreescribirla.

\section{Otra solución}
Para evitar las búsquedas dentro del archivo \texttt{\$HOME/.profile} se puede agregar una sola vez una línea que ejecute un
script en \texttt{\$HOME/.config/} donde están las configuraciones de LXDE. De esta forma, cada vez que se llama a la función
\texttt{set\_xrandr\_info()} se modifica solamente el script y no el archivo \texttt{\$HOME/.profile}.


\section{Casos de prueba}

\section{Trabajo a futuro}

\section{Conclusiones}

\section{Agradecimientos}

\section{Referencias y enlaces de interés}

\begin{enumerate}
\item Página principal de Lihuen GNU/Linux\\
\margin \texttt{http://lihuen.linti.unlp.edu.ar/index.php/Portada}}\\\margin[Visitado el día 9 de enero de 2009]
\end{enumerate}

\end{document}