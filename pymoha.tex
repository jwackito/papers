\documentclass[final,narroweqnarray,inline,twoside]{ieee}

\usepackage{ucs}
\usepackage[utf8x]{inputenc}
\usepackage[spanish]{babel}
\usepackage{fontenc}
\usepackage{graphicx}
\usepackage{textcomp}
% \usepackage{tipa}

\newcommand{\itref}[1]{[{#1}]}
%uso: \itref{1}
%1 es el número de la referencia

\author{Javier Díaz, Claudia Banchoff, Einar Lanfranco, Aldo Vizcaino, Joaquín Bogado
\\LINTI. Facultad de Informática, Universidad Nacional de La Plata.
\\La Plata, B1900ASD, ARGENTINA
}
\title{Implementación de un programa (PyMoHa) parar controlar el mouse mediante el uso de un pulsador para personas con discapacidades motrices graves.}

\hyphenation{}
\begin{document}
\pagenumbering{none}
\maketitle
\sloppy

\begin{abstract}
El abanico de herramientas de accesibilidad para facilitar el acceso a recursos informáticos a personas con discapacidades
motrices graves, es más bien pequeño. Al contrario de lo que pasa en aplicaciones para personas con disminuciones visuales,
la heterogeneidad de las discapacidades motrices obliga a veces a plantear soluciones muy específicas.
Una posible solución de software que abarca una amplia gama de disminuciones motrices es la de controlar el mouse a
través de un pulsador, es decir, un dispositivo cuya única acción sea presionar un botón. Este trabajo describe la creación
de la herramienta PyMoHa, un programa para controlar todas las acciones del mouse mediante el uso de un pulsador.
\end{abstract}

\noindent \textbf{Keywords: } python, pymoha, mouse handler, pulsador, multiplataforma, discapacidades motrices

\section{Introducción}
El abanico de herramientas de accesibilidad para facilitar el acceso a recursos informáticos a personas con discapacidades
motrices graves, es más bien pequeño. Al contrario de lo que pasa en aplicaciones para personas con disminuciones visuales,
la heterogeneidad de las discapacidades motrices obliga a veces a plantear soluciones muy específicas.
Una posible solución de software que abarca una amplia gama de disminuciones motrices es la de controlar el mouse a
través de un pulsador, es decir, un dispositivo cuya única acción sea presionar un botón. Este trabajo describe la creación
de la herramienta PyMoHa, un programa para controlar todas las acciones del mouse mediante el uso de un pulsador. 
\section{Trabajos relacionados}

\section{Otra sección}
%desarrollo, más de una sección.

\section{Casos de prueba}

\section{Trabajo a futuro}

\section{Conclusiones}

\section{Agradecimientos}

\section{Referencias y enlaces de interés}

\begin{enumerate}
\item Página principal de Lihuen GNU/Linux\\
\newblock {\texttt{http://lihuen.linti.unlp.edu.ar/index.php /Portada}}\\[Visitado el día 9 de enero de 2009]
\end{enumerate}

\end{document}