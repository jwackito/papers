\documentclass[final,narroweqnarray,inline,twoside]{ieee}

\usepackage{ucs}
\usepackage[utf8x]{inputenc}
\usepackage[spanish]{babel}
\usepackage{fontenc}
\usepackage{graphicx}
\usepackage{textcomp}
% \usepackage{tipa}

\newcommand{\itref}[1]{[{#1}]}
%uso: \itref{1}
%1 es el número de la referencia

\author{Joaquín Bogado
\\LINTI. Facultad de Informática, Universidad Nacional de La Plata.
\\La Plata, B1900ASD, ARGENTINA
}
\title{}

\hyphenation{}
\begin{document}
\pagenumbering{none}
\maketitle
\sloppy

\begin{abstract}
Muchos de los componentes de impresoras, discos y escaners que se descartan no han llegado al final de su vida util y siguen
funcionando.
Este trabajo trata sobre la reutilización de los motores paso a paso de equipos como los mencionados anteriormente que hayan
sido descartados, para tareas de presición en astronomía aficionada, concretamente para la motorización de la montura de un
telescopio y para el control del foco.
\end{abstract}

\noindent \textbf{Keywords: } 

\section{Introducción}
Los motores de paso, también conocidos como steppers o motores paso a paso (PaP), son comunmente 
\section{Trabajos relacionados}

\section{Otra sección}
%desarrollo, más de una sección.

\section{Casos de prueba}

\section{Trabajo a futuro}

\section{Conclusiones}

\section{Agradecimientos}

\section{Referencias y enlaces de interés}

\begin{enumerate}
\item Página principal de Lihuen GNU/Linux\\
\newblock {\texttt{http://lihuen.linti.unlp.edu.ar/index.php /Portada}}\\[Visitado el día 9 de enero de 2009]
\end{enumerate}

\end{document}