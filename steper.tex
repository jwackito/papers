\documentclass[final,narroweqnarray,inline,twoside]{ieee}

\usepackage{ucs}
\usepackage[utf8x]{inputenc}
\usepackage[spanish]{babel}
\usepackage{fontenc}
\usepackage{graphicx}
\usepackage{textcomp}
% \usepackage{tipa}

\newcommand{\itref}[1]{[{#1}]}
%uso: \itref{1}
%1 es el número de la referencia

\author{Joaquín Bogado
\\LINTI. Facultad de Informática, Universidad Nacional de La Plata.
\\La Plata, B1900ASD, ARGENTINA
}
\title{Reutilización de motores de paso para motorización de una montura equatorial de telescopio}

\hyphenation{}
\begin{document}
\pagenumbering{none}
\maketitle
\sloppy

\begin{abstract}
Muchos de los componentes de impresoras, discos y escaners que se descartan no han llegado al final de su vida util y siguen
funcionando.
Este trabajo trata sobre la reutilización de los motores paso a paso de equipos como los mencionados anteriormente que hayan
sido descartados, para tareas de precisión en astronomía aficionada, concretamente para la motorización de una montura
equatorial de un telescopio y para el control del foco del telescopio.
\end{abstract}

\noindent \textbf{Keywords: } motor de paso, 

\section{Introducción}
Los motores de paso, también conocidos como steppers o motores paso a paso (PaP), son comúnmente usados en impresoras para
mover los inyectores y cabezales de impresión o en escáneres para mover la lámpara de escaneo. La particularidad de estos
motores es que sus movimientos son discretos y repetibles en una determinada cantidad de grados. Cada paso se produce por la
activación de las bobinas del motor que funcionan como electroimanes, atrayendo uno de los polos del rotor. Esto permite
controlar el avance o retroceso del motor en forma precisa activando o desactivando las bobinas mediante pulsos controlados
mediante un programa.
\section{Trabajos relacionados}
Los sistemas de motorización de monturas para telescopio comerciales que se consiguen en el país rondan los 350 \$AR. Esto
es solo para motorizar la montura con el fin de contrarrestar el movimiento de rotación terrestre, ya que los sistemas
motorizados de guiado (comúnmente conocidos como GOTO) son mucho más caros. Un enfocador electrónico ronda al rededor de los
320 \$AR. El kit GOTO permite seleccionar el punto en el cielo que se desea observar, entrando un numero de catálogo que hace
referencia a una base de datos dentro del dispositivo. Un GOTO típico tiene una base de datos con entre 30000 y 60000 objetos
celestes aunque la mayoría permite agregar objetos por Internet.
\section{Los motores de paso}
Los motores de paso suelen utilizarce en impresoras para mover los cabezales de impresión, en escáneres para mover la
lámpara, en lectoras y grabadoras de CDs y DVDs para mover el cabezal laser y en ejes de discos rígidos para mover los
platos, entre otros usos.

Un motor de paso es un motor eléctrico sin escobillas 
A diferencia de los motores de corriente continua o de los de corriente alterna, los motores de paso hacen girar el eje en
forma discreta en una cantidad de grados igual para cada paso. Así, un motor de 1,8 grados por paso tendrá 200 pasos por
revolución.



\section{Problemas con los motores}
\section{La plataforma Arduino}
\section{El foco}
\section{La montura}
\section{Casos de prueba}

\section{Trabajo a futuro}

\section{Conclusiones}
\section{Glosario}
\begin{enumerate}
 \item[\texttt{escobilla}:] Un dispositivo que conduce la corriente entre alambres estacionarios y partes móviles, usado
comúnmente en motores, generadores y alternadores eléctricos. Conocidas en inglés como \textit{brushes}.
 \item[\texttt{estator}:] Una parte fija o estacionaria al rededor de la cual gira un rotor. En inglés, \textit{stator}.
 \item[\texttt{rotor}:] La parte que gira de una máquina eléctrica.
\end{enumerate}

\section{Agradecimientos}

\section{Referencias y enlaces de interés}
\begin{enumerate}
\item Página principal de Lihuen GNU/Linux\\
\newblock {\texttt{http://lihuen.linti.unlp.edu.ar/index.php /Portada}}\\[Visitado el día 9 de enero de 2009]
\end{enumerate}

\end{document}