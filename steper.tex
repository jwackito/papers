\documentclass[final,narroweqnarray,inline,twoside]{ieee}

\usepackage{ucs}
\usepackage[utf8x]{inputenc}
\usepackage[spanish]{babel}
\usepackage{fontenc}
\usepackage{graphicx}
\usepackage{textcomp}
% \usepackage{tipa}

\newcommand{\itref}[1]{[{#1}]}
%uso: \itref{1}
%1 es el número de la referencia

\author{Javier Díaz, Claudia Banchoff, Einar Lanfranco, Joaquín Bogado
\\LINTI. Facultad de Informática, Universidad Nacional de La Plata.
\\La Plata, B1900ASD, ARGENTINA
}
\title{Reutilización de motores de paso para motorización de una montura ecuatorial de telescopio}

\hyphenation{}
\begin{document}
\pagenumbering{none}
\maketitle
\sloppy

\begin{abstract}
Muchos de los componentes de impresoras, discos y escáneres que se descartan no han llegado al final de su vida útil y
siguen funcionando.
Este trabajo trata sobre la reutilización de los motores paso a paso de equipos como los mencionados anteriormente que hayan
sido descartados, para tareas de precisión en astronomía aficionada, concretamente para la motorización de una montura
ecuatorial de un telescopio y para el control del foco del telescopio.
\end{abstract}

\noindent \textbf{Keywords: } motor de paso, arduino, freeduino, reutilización de residuos tecnológicos

\section{Introducción}
Los motores de paso, también conocidos como steppers o motores paso a paso (PaP), son comúnmente usados en impresoras para
mover los inyectores y cabezales de impresión o en escáneres para mover la lámpara de escaneo. La particularidad de estos
motores es que sus movimientos son discretos y repetibles en una determinada cantidad de grados. Cada paso se produce por la
activación de las bobinas del motor que funcionan como electroimanes, atrayendo uno de los polos del rotor. Esto permite
controlar el avance o retroceso del motor en forma precisa activando o desactivando las bobinas mediante pulsos controlados
mediante un programa.
\section{Trabajos relacionados}
Los sistemas de motorización de monturas para telescopio comerciales que se consiguen en el país rondan los 350 \$AR. Esto
es solo para motorizar la montura con el fin de contrarrestar el movimiento de rotación terrestre, ya que los sistemas
motorizados de guiado (comúnmente conocidos como GOTO) son mucho más caros. Un enfocador electrónico ronda alrededor de los
320 \$AR. El kit GOTO permite seleccionar el punto en el cielo que se desea observar, entrando un número de catálogo que hace
referencia a una base de datos dentro del dispositivo. Un GOTO típico tiene una base de datos con entre 30000 y 60000 objetos
celestes, aunque la mayoría permite agregar objetos por Internet.
\section{Los motores de paso}
Los motores de paso suelen utilizarse en impresoras para mover los cabezales de impresión, en escáneres para mover la
lámpara, en lectoras y grabadoras de CDs y DVDs para mover el cabezal láser y en ejes de discos rígidos para mover los
platos, entre otros usos.

Un motor de paso es un motor eléctrico sin escobillas, sincrónico, que divide la rotación de su eje o rotor en un determinado
número de pasos\itref{1}. A diferencia de los motores de corriente continua o de los de corriente alterna, los motores de
paso hacen girar el eje en forma discreta en una cantidad de grados igual para cada paso. Así, un motor de 1,8 grados por
paso tendrá 200 pasos por revolución.

Internamente, los motores de paso poseen electroimanes dentados dispuestos alrededor de un rotor con una rueda de hierro u
otro material magnético, también dentada. Los electroimanes son energizados en una determinada secuencia mediante un
circuito generador de pulsos o un microcontrolador. Para hacer que el eje del motor gire en un determinado sentido, uno de
los electroimanes es activado por un pulso, atrayendo hacia sus dientes a los dientes del rotor. Cuando los dientes del
rotor se alinean con respecto al primer electroimán, quedan ligeramente desalineados con respecto al siguiente electroimán.
Así, cuando se activa el segundo electroimán y se desactiva el primero, el rotor gira levemente para alinearse con los
dientes del segundo electroimán, produciendo como resultado un paso.

Las características principales de un motor de paso son:
\begin{enumerate}
 \item Los motores de paso, mientras están en funcionamiento, necesitan permanentemente ser alimentados con electricidad.
Comúnmente funcionan con entre 5 y 24 V.
 \item A medida que la velocidad de rotación aumenta, el torque disminuye. La mayoría de los motores exhiben su mayor torque
cuando están conectados a una fuente de alimentación eléctrica y están detenidos. Claro que un motor con mucho torque
detenido no es de mucha utilidad: el torque es de mayor utilidad cuando el motor está girando.
 \item Es posible aumentar el torque del motor limitando la corriente y aumentando el voltaje que entra al motor. Este tipo
de circuitos se conoce como \textit{chopper circuit}. Existen varias soluciones comerciales que implementan estos circuitos.
 \item Los motores de paso vibran más que otros tipos de motores. Esto se debe a que un paso suele hacer saltar el eje de
una posición a otra y el movimiento no es continuo.
 \item Esta vibración es importante debido a que a ciertas velocidades el motor puede perder torque o incluso cambiar de
dirección.
 \item Existen técnicas que mitigan los problemas producidos por las vibraciones y la resonancia.
 \item Los motores con un número grande de fases exhiben una operación más suave que aquellos con pocas fases.
\end{enumerate}
Internamente y según la fabricación del rotor, los motores de paso se dividen en motores de imán permanentemente (PM), de
reluctancia variable (VR) o híbridos. Los motores de imán permanentemente tienen en su rotor un imán y opera atrayendo o
repeliendo los dientes del rotor hacia los electroimanes del estator. Los motores de reluctancia variable tienen un rotor de
hierro y se basan en el principio de que la mínima reluctancia se produce en un mínimo espacio. Por lo tanto los puntos del
rotor son atraidos hacia los electroimanes del estator. Los motores híbridos usan una combinación de las dos técnicas para
obtener máxima potencia en un espacio de empaquetado más reducido.

Para los motores de dos fases, que comúnmente se encuentran en hardware obsoleto, existen dos disposiciones de bobinado para
los electroimanes. De esta manera, se los puede clasificar en motores unipolares o bipolares. Cabe recordar que la fuerza
del campo magnetico ejercida por el electroimán del las bobinas depende directamente del largo del alambre de cobre, de la
cantidad de vueltas de la bobina y de la cantidad de corriente con la que se alimenta al motor.
Existe fórmulas precisas para calcular la intensidad del campo mágnetico de una bobina, pero al alcance del trabajo
satizface saber que a mayor cantidad de alambre en una bobina y a mayor voltaje de alimentación de la bobina,
mayor será el campo magnético generado, lo que se traduce en un mayor torque del motor.

Las configuraciones ángulos típicas de cada paso son 0,72°; 1,8°; 3,6°; 7,5°; 15°, 45° y hasta 90° aunque se pueden obtener
pasos más pequeños usando "medios pasos" (half step) o "micropasos" (microstep).

\section{Los motores unipolares}
Los motores unipolares tienen dos bobinados por fase, uno por cada dirección del campo magnético. Esto hace a los motores
unipolares muy fáciles de controlar, debido a que el cambio de polo de una bobina puede hacerse sin invertir la dirección de
la corriente. En circuitos para controlar los motores unipolares, generalmente alcanza con un transistor por bobina.

Uno de los extremos del bobinado de cada fase se denomina cable común (el extremo que se conecta al
polo positivo a entre 5 y 12 V) y es compartido por todas las bobinas de la fase. Esta configuración muestra tres cables por
fase, es decir, seis cables para un motor de dos fases. Generalmente, los cables comunes de cada fase están unidos
internamente, por lo que no es difícil encontrar motores unipolares con cinco cables en lugar de seis.
\begin{figure}[h]
 \centering
 \includegraphics{./unipolar.png}
 % unipolar.svg: 249x261 pixel, 72dpi, 8.78x9.21 cm, bb=0 0 249 261
 \caption{Esquema de conexionado de un motor unipolar}
 \label{fig: Esquema de conexionado de un motor unipolar}
\end{figure}

Un detalle importante a tener en cuenta cuando se trabaja con motores reciclados (los cuales generalmente no cuentan con
hojas de datos) es como averiguar cual es el cable común y cuales pertenecen a que bobina.

\section{Los motores bipolares}
Los motores bipolares tienen un bobinado por fase en lugar de dos, como tienen los motores unipolares. En este tipo de
motores, por lo tanto, para invertir la orientación del campo magnético es necesario invertir la dirección de la corriente
eléctrica que pasa por la bobina. Esto hace que los circuitos para controlar este tipo de motores sean más complejos. Estos
motores presentan generalmente cuatro cables, dos por fase, ninguno común.

Como se usa solamente un bobinado por fase (y no dos como en los motores unipolares), la cantidad de alambre de cada bobina
es menor. Así, un motor bipolar con igual cantidad de alambre en cada bobina que un motor unipolar generará un campo
magnético mayor dada una misma intensidad de corriente eléctrica, debido a que en un motor unipolar se usa solamente la mitad
de la bobina para inducir una dirección del campo magnético.

\begin{figure}[h]
 \centering
 \includegraphics{./bipolar.png}
 % bipolar.svg: 249x261 pixel, 72dpi, 8.78x9.21 cm, bb=0 0 249 261
 \caption{Esquema de conexionado de un motor bipolar}
 \label{fig: Esquema de conexionado de un motor bipolar}
\end{figure}

\section{Problemas con los motores}
Uno de los principales inconvenientes al tratar con motores de paso extraídos de hardware obsoleto es la falta de
documentación asociada a dichos motores. No es trivial por ejemplo, aislar los cables para identificar cuál activa qué
bobina.
Una técnica para aisalar los cables 
\section{La plataforma Arduino}
Arduino es una plataforma de hardware y software con especificaciones libres para prototipos de dispositivos electrónicos.
Debido a que está orientada a aficionados, diseñadores y artistas, resulta fácil de usar y aprender. Posee una comunidad
bastante activa que permanentemente aporta diseños y soluciones nuevas. La plataforma básica cuenta con una plaqueta de pbc
a la que vienen soldados los componentes, entre los que se cuentan, dependiendo de la versión de arduino, un conector USB o
Serial y un microcontrolador ATMega8, ATMega168, ATMega328 o ATMega644 de arquitectura AVR. El microcontrolador es programado
usando el Lenguaje de Programación Arduino, por lo que es necesario descagar el IDE de Arduino para poder escribir, compilar
y almacenar los programas en los microcontroladores ATMega. Este IDE hace uso del compilador gcc-avr, también libre, para
hacer compilación cruzada (cross compiling) para arquitecturas AVR. 
\begin{figure}[h]
 \centering
 \includegraphics{./arduino-ide.png}
 % arduino-ide.png: 320x384 pixel, 96dpi, 8.47x10.16 cm, bb=0 0 240 288
 \caption{Interfaz de programacion Arduino}
 \label{fig: Interfaz de programacion Arduino}
\end{figure}


El Lenguaje de Programación Arduino está basado en C++ y cuenta con una gran variedad de librerías y funciones predefinidas
de alto nivel, muy bien documentadas, que hacen su aprandisaje relativamente sencillo. Como ejemplo podemos usar la librería
usada en este trabajo, llamada \texttt{stepper}. Solo cuenta con cuatro funciones:
\begin{itemize}
 \item \texttt{Stepper(steps, pin1, pin2)}: Crea una instacia de la clase \texttt{Stepper} que hace referencia a un motor en
particular adosado al circuito Arduino. El parámetro \texttt{steps} hace referencia a la cantidad de pasos por revolución
del motor en cuestion, mientras que \texttt{pin1} y \texttt{pin2}, son los pines al los cuales estan conectados los dos
cables de control del motor.
 \item \texttt{Stepper(steps, pin1, pin2, pin3, pin4)}: Ídem a la anterior, solo que para un motor con 4 cables de control.
 \item \texttt{setSpeed(rpm)}: Indica las revoluciones por minuto para una instancia de la clase Stepper.
 \item \texttt{step(steps)}: Mueve el motor \texttt{steps} pasos. Si es negativo se mueve en la dirección contraria. La
dirección depende del conexionado de cada motor.
\end{itemize}



\section{El foco}
\section{La montura}
\section{Casos de prueba}
\section{Trabajo a futuro}
\section{Conclusiones}
\section{Glosario}
\begin{enumerate}
 \item[\texttt{escobilla}:] Un dispositivo que conduce la corriente entre alambres estacionarios y partes móviles, usado
comúnmente en motores, generadores y alternadores eléctricos. Conocidas en inglés como \textit{brush}.
 \item[\texttt{estator}:] Una parte fija o estacionaria al rededor de la cual gira un rotor. En inglés, \textit{stator}.
 \item[\texttt{rotor}:] La parte que gira en torno a un eje en una máquina eléctrica.
 \item[\texttt{torque}:] Fuerza que puede hacer el motor antes de girar en falso.
\end{enumerate}

\section{Agradecimientos}

\section{Referencias y enlaces de interés}
\begin{enumerate}
\item Stepper motor\newline
\margin \texttt{http://en.wikipedia.org/wiki/Stepper\_motor}\\\margin [Visitado el día 29 de junio de 2010]
\end{enumerate}
\end{document}