\documentclass[final,narroweqnarray,inline,twoside]{ieee}
%\documentclass[a4paper,10pt]{article}

\usepackage{ucs}
\usepackage[utf8x]{inputenc}
\usepackage[spanish]{babel}
\usepackage{fontenc}
\usepackage{graphicx}
\usepackage{textcomp}
% \usepackage{tipa}

\newcommand{\itref}[1]{[{#1}]}
%uso: \itref{1}
%1 es el número de la referencia

\author{Francisco Javier Díaz \\Joaquín Ignacio Bogado García \\
Claudia Banchoff Tzancoff\\Einar Felipe Lanfranco \\ \{javierd,jbogado,cbanchoff,einar\}@linti.unlp.edu.ar
\\LINTI - Facultad de Informática - Universidad Nacional de La Plata
\\La Plata, B1900ASD, ARGENTINA}
\title{Lethe\\Modificaciones realizadas para usar particiones como pseudo-tmpfs}

\hyphenation{Rootaufs SourceForge}
\begin{document}
\pagenumbering{none}
\maketitle
\sloppy

\begin{abstract}
\noindent Lethe es una herramienta que permite congelar el contenido del sistema de archivos de sistemas basados en Debian
GNU/Linux mediante la utilización de AUFS y tmpfs. Su funcionalidad es similar a la implementada por herramientas como Deep
Freeze o Windows Steady State en ambientes Microsoft.

Este informe describe una serie de modificaciones realizadas al programa Lethe para ampliar la cantidad de cambios
que puedan hacerse al sistema de archivos y que estos no dependan del tamaño de la memoria RAM, evitando así fallos en
el sistema por agotamiento de recursos.
\end{abstract}

\noindent \textbf{Palabras clave: } Lethe, Lihuen, GNU/Linux, Debian, Aufs, Deep Freeze, Steady State

\section{Contexto}
Este trabajo se realizó en el marco del proyecto Lihuen GNU/Linux. El Objetivo del proyecto es desarrollar y mantener una
distribución de GNU/Linux, actualmente desarrollada por un equipo de docentes y alumnos del Laboratorio de Investigación de
Nuevas Tecnologías Informáticas (LINTI) de la Universidad Nacional de La Plata. La distribución esta orientada a entornos
educativos y cuenta con un gran número de herramientas divididas en áreas como geografía, química, música, nivel inicial,
matemáticas, lengua o astronomía. Además incluye herramientas para la mayoría de las materias que se dictan en la Facultad
de Informática de la UNLP. Parte de los objetivos del proyecto incluyen la utilización, formación y difusión en materia de
software libre. 

\section{Introducción}
En ambientes con equipos de uso público como laboratorios, salas de computación, cibercafés o academias, es deseable que el
uso por parte de un usuario no afecte la configuración ni el software instalado en ese equipo.

La Facultad de Informática cuenta con tres salas de computación para alumnos, donde las máquinas poseen doble booteo:
sistemas Microsoft Windows XP y Lihuen GNU/Linux, que se desarrolla en la Facultad de Informática de la UNLP y se encuentra
instalado en más de 60 equipos de dicha institución y en otras salas de computación de la Provincia de Buenos Aires. Estas
salas tienen un uso académico y, en algunos casos, se requiere que los usuarios puedan instalar y probar herramientas durante
las clases.

Esta situación se puede generalizar a otros ambientes con equipos de uso público, como laboratorios, cibercafés o gabinetes
de escuelas, en donde es deseable que las acciones de un usuario no afecten la configuración ni el software instalado en ese
equipo. Esto impide que los cambios que realice afecten a futuros usuarios del mismo equipo.

Lethe es sistema que permite proteger el contenido del o los discos de una PC independientemente de los cambios que realice
un usuario. Dicha protección se realiza mediante el congelamiento de las particiones del disco utilizando Aufs, está pensado
y desarrollado para entornos Lihuen GNU/Linux \itref{1}, y en general, es aplicable en cualquier sistema derivado de Debian
GNU/Linux\itref{2}.

\section{El funcionamiento de Lethe}
El desarrollo de Lethe se basa en una generalización del código de RootAUFS\footnote
{https://help.ubuntu.com/community/aufsRootFileSystemOnUsbFlash}, escrito originalmente por Nicholas Schembri. RootAUFS fue
pensado originalmente para reducir el impacto de los ciclos de escritura en las memorias de estado sólido (que reemplazan al
disco rígido en algunas portátiles y \textit{netbooks}) escribiendo los cambios realizados al sistema de archivos en memoria
RAM en lugar de usar la memoria de estado sólido. Para esto, es necesario crear un sistema de archivos en memoria RAM llamado
tmpfs y unificarlo con la raíz del sistema de archivos de la memoria de estado sólido mediante
Aufs\footnote{http://aufs.sourceforge.net/}.

Aufs fue en un primer momento una reimplementación de Unionfs\footnote{http://www.filesystems.org/project-unionfs.html}
versión 1.x y, aunque ahora se ha vuelto un proyecto totalmente distinto, conserva las características principales de
Unionfs. Básicamente, es un sistema de archivos \textbf{unificable} y \textbf{apilable}; unificable porque el contenido de
varios directorios (\textit{branches} o ramificaciones) tendrá la apariencia de uno solo mientras que el contenido físico de
estos estará separado; apilable porque los \textit{branches} se apilan uno sobre otro en el orden en que se van montando.
Cada \textit{branch} puede tener sus propios atributos; por ejemplo, si el \textit{branch} es de sólo lectura o de
lectura/escritura, si tiene control de errores o no, etc.

Lethe se aprovecha de la capacidad de Aufs para permitir el acceso a múltiples directorios a través de un único punto de
montaje.

El sistema de archivos tmpfs, por su parte, es un sistema de archivos sumamente sencillo, que utiliza la memoria RAM (toda la
disponible o parte de esta, dependiendo de los parámetros pasados a la hora de montar la unidad) como si fuera un sistema de
archivos en disco. Además, tmpfs comparte el espacio de la memoria RAM con otros sistemas tmpfs (por ejemplo \textit{udev} y
\textit{shm}).

\section{El problema}
El problema radica en que tmpfs no utiliza el espacio de almacenamiento de los discos si estos existen. No es
posible crear un sistema de archivos de tamaño mayor que el total de memoria RAM disponible (generalmente de 2 a 4 GB). 
Por lo tanto, cuando la memoria se termina, por ejemplo debido a la copia de uno o varios archivos de gran tamaño (o incluso
el borrado de varios archivos), el sistema se vuelve inestable debido a la falta de memoria, se torna inoperable y es
necesario reiniciarlo.

\section{\textit{Internals}}
Una parte importante en el funcionamiento de Lethe es la que crea el sistema de archivos temporal en la carpeta
\texttt{/rw}. Esta carpeta es la que posteriormente se une a la raíz del sistema de archivos mediante el empleo de Aufs y
sobre la cual se hacen los cambios de todos los puntos de montaje que se congelan.
La línea responsable de montar esta carpeta es la siguiente:
\\\indent\texttt{run\_echo "mount -t tmpfs aufs-tmpfs /rw"}

La función \texttt{run\_echo} ejecuta el comando que recibe como argumento, realiza un manejo precario de errores y aborta la
ejecución del \textit{script} si el comando falla. Si el comando es exitoso, todo lo que se escriba en la carpeta
\texttt{/rw} será
volátil ya que el sistema de archivos tmpfs usa exclusivamente memoria RAM como medio de almacenamiento (como ya se
ha mencionado).

En este punto, el sistema raíz está montado en la variable \texttt{\${rootmnt}}. Luego se mueve esta raíz a \texttt{/ro}
con el comando siguiente:
\\\indent\texttt{run\_echo "mount --move \${rootmnt} /ro"}

De esta manera el sistema de archivos temporal queda montado en \texttt{/rw} y el sistema de solo lectura queda montado en
\texttt{/ro}.

Posteriormente ambos sistemas de archivos se unifican con Aufs. El parámetro \texttt{dirs} indica que tanto \texttt{/rw} como
\texttt{/ro} van a ser accesibles a través de \texttt{/aufs} y que \texttt{/ro} además va a ser de solo lectura
(\texttt{/ro=ro}). 
\\\indent\texttt{run\_echo "mount -t aufs -o dirs=/rw:/ro=ro aufs /aufs"}

Debido a esto, ambos sistemas de archivos quedan ocultos en \texttt{/aufs/rw} y \texttt{/aufs/ro}. Para que puedan ser
accedidos como un único sistema de archivos a través de \texttt{/} es necesario mover \texttt{/aufs} a \texttt{/root} (la
cual en posteriores \textit{scripts} de la secuencia de arranque se transforma en \texttt{/}).
\\\indent\texttt{run\_echo "mount --move /aufs /root"}

Básicamente\footnote{Se han evitado algunos detalles de implementación como la desactivación de \textit{AppArmor}
para Ubuntu, correcciones a \texttt{/etc/mtab} y algunas tareas que se agregan a \texttt{rc.local} con el fin de simplificar
la comprensión de este texto y no desviarnos del objetivo del trabajo.}, esto permite que Lethe funcione, ya que cuando se
crea un archivo en \texttt{/} o cualquier otro subdirectorio, se crea en realidad en \texttt{/aufs/rw/}. Por ejemplo, si
ejecutamos \texttt{touch /home/lihuen/miarchivo}, además de ser accesible mediante \texttt{/home/lihuen/miarchivo} será
accesible mediante \texttt{/aufs/rw/home/lihuen/miarchivo}. Sin embargo, si tratamos de acceder al sistema de archivos
original mediante \texttt{/aufs/ro}, por ejemplo ejecutando \texttt{rm /aufs/ro/bin/bash} recibiremos un error indicando que
el sistema de archivos es de solo lectura (aún teniendo privilegios de superusuario).

\section{La solución}
Habiendo explicado el funcionamiento interno de Lethe, se describe a continuación una solución al problema del agotamiento
de memoria producido por el uso del sistema de archivos tmpfs.

En lugar de utilizar la (a veces escasa) memoria RAM, podemos dedicar espacio en disco para escribir los cambios producidos
en el sistema de archivos. La idea es tener una partición de tamaño suficiente (y mayor que la memoria RAM) de, digamos, 50 a
100 GB, sin contenido y formateada con algún sistema de archivos (por ejemplo \textit{ext3}). Esta partición del disco debe
tener la etiqueta ´lethe´ para que el instalador la encuentre y configure la variable de entorno \texttt{\$LETHE\_PART} en
\texttt{/etc/lethe/lethe.conf}. En cada reinicio del sistema, si Lethe está activo, borrará el contenido de esta partición y
la utilizará para montar \texttt{/rw} en lugar de montar \texttt{/rw} con el sistema de archivos tmpfs.

La solución se implementó como sigue:
\\\texttt{source \$\{rootmnt\}/etc/lethe/lethe.conf}
\\\texttt{if [ -n "\$LETHE\_PART" ]; then}
\\\texttt{\indent run\_echo "mount \$LETHE\_PART /rw"}
\\\texttt{\indent run\_echo "rm -r -f /rw/*"}
\\\texttt{else}
\\\texttt{\indent run\_echo "mount -t tmpfs aufs-tmpfs /rw"}
\\\texttt{fi}

De esta manera, si la partición existe se utiliza como sistema de archivos temporal y se borra cada vez. Si la variable
\texttt{\$LETHE\_PART} no está declarada, entonces se utiliza el sistema de archivos tmpfs como en el caso anterior.

\section{Ventajas y desventajas}
Esta solución tiene la desventaja de que incrementa el tiempo de arranque de la máquina debido a que durante el arranque se
borra el contenido de la partición dedicada. Esto es necesario para simular el efecto de reiniciar en la memoria RAM. El
tiempo de arranque no es constante entre reinicios ya que depende de la cantidad y el tamaño de los archivos generados en la
sesión previa. 

Otra desventaja es que el descubrimiento de la partición dedicada se realiza durante la instalación, es decir, una vez. Si la
partición cambia será necesario cambiar a mano el archivo \texttt{/etc/lethe/lethe.conf} y apuntar la variable \$LETHE\_PART
al dispositivo que corresponda.

Además es necesario cambiar el esquema de particionamiento para que Lethe haga uso exclusivo de una partición dedicada.

A pesar de lo arriba mencionado, es importante destacar que esta solución permite superar el límite de tamaño en tmpfs,
restringido al tamaño de la memoria RAM, utilizando una partición
dedicada tan grande como la partición utilizada. Además, si la partición no está disponible, el sistema usa el sistema tmpfs
convencional basado en memoria RAM.

\section{Pruebas realizadas}
Se realizaron pruebas exitosas sobre máquinas virtuales donde se generaron archivos nuevos de hasta 6 veces el tamaño de la
memoria RAM sin notar degradaciones en el sistema. Además se realizó la instalación masiva de la nueva versión en una de las
salas de PCs de la Facultad de Informática de la UNLP donde podrá monitorearse su desempeño durante los próximos 6
meses.

Una vez que el paquete esté disponible para el público en general, se espera recibir reportes de usuarios de diferentes
partes del mundo y sobre otros sistemas operativos además de Lihuen, como ya ha sucedido con lanzamientos anteriores.

\section{Trabajo a futuro}
Queda pendiente la actualización de los repositorios Git de SourceForge.net, además de la publicación de los binarios para
el público en general, programada para los próximos días.

Más adelante, se piensa incluir la detección de la partición en tiempo de arranque, permitiendo también la posibilidad de
utilizar \textit{pendrives} en lugar de particiones como pseudo-tmpfs.

Además se realizará una investigación acerca del uso de archivos como dispositivos de bloques. Es sabido que QEMU/KVM utiliza
un sistema que permite crear discos virtuales en archivos que aumentan de tamaño en la medida que sea necesario. Esto
evitaría tener que crear un archivo de, digamos, 50 GB, darle formato y borrarlo en cada reinicio. Además, haría innecesario
el reparticionamiento del disco.

\section{Formación de recursos humanos}
Actualmente en el proyecto Lihuen GNU/Linux trabaja un grupo de 2 docentes y 7 alumnos de la Facultad de Informática de la
UNLP, investigando sobre diversos temas relacionados con la adaptación de Lihuen a las escuelas de la zona. Dentro de estas
áreas se incluyen tópicos como accesibilidad, clientes livianos, herramientas para educación, sistemas operativos. El grupo
también realiza actividades de extensión. Durante el año 2010 se visitaron 5 escuelas de la zona, donde se dieron charlas
acerca de los beneficios del uso de software libre a cursos de grados superiores de secundario. El grupo presento varios
trabajos en las pasadas Jornadas Regionales de Software Libre\itref{22} realizadas en San Luis a finales de octubre de 2010.
Dentro de los trabajos presentados, se incluyen un taller de programación de interfaces gráficas con PyGKT, una charla sobre
el uso de robots en la enseñanza de programación en escuelas técnicas y una charla de Lethe donde se abarcaron los temas de
funcionamiento e implementación.

\section{Referencias y enlaces de interés}

\begin{enumerate}
\item Página principal de Lihuen GNU/Linux\\
\newblock {\texttt{http://lihuen.linti.unlp.edu.ar/index.php /Portada}}
\item Página principal de Lethe\\
\newblock {\texttt{http://sourceforge.net/projects/lethe/}}
\item Javier Díaz, Joaquín Bogado, Claudia Banchoff, Einar Lanfranco
\\\textit{Lethe: Congelador de particiones para Lihuen GNU/Linux}\\
\newblock {\textit{Paper }publicado en el Encuentro Chileno de Ciencias de la Computación, Santiago de Chile, Chile -
nov/2009}
\item AUFS (Another Union File System) Root File System On USB Flash - Community Ubuntu Documentation\\
\texttt{https://help.ubuntu.com/community/aufsRootFile SystemOnUsbFlash}
\item Unionfs: A Stackable Unification File System\\
\texttt{http://www.filesystems.org/project-unionfs.html}
\item Jornadas Regionales de Software Libre - San Luis (Oct 2010)\\
\texttt{http://jornadasregionales.org/jrsl2010v2/schedule/index}
\end{enumerate}

\end{document}